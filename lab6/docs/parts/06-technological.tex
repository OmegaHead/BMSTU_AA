\chapter{Технологическая часть}

В данной главе описаны средства реализации, приведены листинги кода и функциональные тесты.

\section{Средства реализации}

Для разработки данной лабораторной работы был выбран язык программирования Swift \cite{swift}. 
Этот выбор обусловлен возможностью измерения процессорного времени \cite{cpu-time-measure} и соответствием с выдвинутыми техническими требованиям.

Измерение времени выполнения алгоритмов производится с использованием функции \textit{clock\_gettime()} \cite{cpu-time-measure}.
\clearpage
\section{Сведения о модулях программы}

Программа разбита на следующие модули:

\begin{itemize}
	\item \texttt{main.swift} --- точка входа в программу, где происходит вызов алгоритмов через интерфейс;
	\item \texttt{Algorithms.swift} --- содержит реализации методов решения задачи коммивояжёра (полным перебором и с использованием муравьиного алгоритма);
	\item \texttt{CPUTimeMeasure.swift} --- измеряет время работы алгоритмов с учетом заданного количества повторений;
	\item \texttt{GraphRenderer.swift} --- Строит графики для каждого из алгоритмов с учетом заданного количества повторений для каждого алгоритма;
\end{itemize}

\section{Реализация алгоритмов}

В листингах \ref{lst:bruteforce} -- \ref{lst:updatePheromones} приведены реализации методов решения задачи коммивояжёра: полным перебором (листинги \ref{lst:bruteforce} -- \ref{lst:next_permutation}) и с использованием муравьиного алгоритма (листинги \ref{lst:ant} -- \ref{lst:updatePheromones}).

В листинге \ref{lst:input} приведена реализация ввода матрицы.

\clearpage

\lstinputlisting[label=lst:bruteforce,caption=Полный перебор, firstline=20,lastline=36]{../src/AA6/Sources/AA6/Algorithms.swift}
\lstinputlisting[label=lst:calculatePathDistance,caption=Функция вычисления длины маршрута, firstline=38,lastline=45]{../src/AA6/Sources/AA6/Algorithms.swift}
\clearpage
\lstinputlisting[label=lst:next_permutation,caption=Функция получения нового маршрута, firstline=48,lastline=75]{../src/AA6/Sources/AA6/Algorithms.swift}
\clearpage

\lstinputlisting[label=lst:ant,caption=Муравьиный алгоритм, firstline=119,lastline=153]{../src/AA6/Sources/AA6/Algorithms.swift}
\lstinputlisting[label=lst:generateAntPath,caption=Функция построения пути муравья, firstline=155,lastline=172]{../src/AA6/Sources/AA6/Algorithms.swift}
\clearpage
\lstinputlisting[label=lst:selectNextCity,caption=Функция выбора следующего города, firstline=174,lastline=205]{../src/AA6/Sources/AA6/Algorithms.swift}
\clearpage
\lstinputlisting[label=lst:depositPheromone,caption=Функция работы с элитным муравьем, firstline=216,lastline=253]{../src/AA6/Sources/AA6/Algorithms.swift}
\clearpage
\lstinputlisting[label=lst:updatePheromones,caption=Функция обновления феромонов, firstline=225,lastline=238]{../src/AA6/Sources/AA6/Algorithms.swift}
\clearpage
\lstinputlisting[label=lst:input,caption=Функция ввода матрицы, firstline=8,lastline=34]{../src/AA6/Sources/AA6/main.swift}


\section{Функциональные тесты}

В таблице приведены функциональные тесты для методов решения задачи коммивояжёра. 
Все тесты были успешно пройдены.

\clearpage


\begin{table}
	\caption{Тесты для умножения матриц}
	\begin{center}
	\begin{tabular}[c]{|c|c|}
		\hline
		Матрица смежности & Ожидаемый результат \\
		\hline
		$ \begin{pmatrix}
			0 &  4 &  2 &  1 & 7 \\
			4 &  0 &  3 &  7 & 2 \\
			2 &  3 &  0 & 10 & 3 \\
			1 &  7 & 10 &  0 & 9 \\
			7 &  2 &  3 &  9 & 0
		\end{pmatrix}$ &
		15, [0, 2, 4, 1, 3, 0] \\
		
		$ \begin{pmatrix}
			0 & 1 & 2 \\
			1 & 0 & 1 \\
			2 & 1 & 0	
		\end{pmatrix}$ &
		4, [0, 1, 2, 0] \\
		
		$ \begin{pmatrix}
			0 & 15 & 19 & 20 \\
			15 &  0 & 12 & 13 \\
			19 & 12 &  0 & 17 \\
			20 & 13 & 17 &  0
		\end{pmatrix}$ &
		64, [0, 1, 2, 3, 0] \\
		\hline
	\end{tabular}
\end{center}
	\end{table}

\section*{Вывод}

В данной главе были представлены требования к программному обеспечению, описаны средства реализации, приведены листинги кода алгоритмов и функциональные тесты, подтверждающие корректность работы алгоритмов.
