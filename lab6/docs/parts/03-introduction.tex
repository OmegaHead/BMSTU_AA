\chapter*{Введение}
\addcontentsline{toc}{chapter}{Введение}

В данной лабораторной работе будут рассмотрены методы решения задачи коммивояжера.

Одной из наиболее известных и значимых задач в области транспортной логистики является задача коммивояжёра, также известная как "задача о странствующем торговце". 
Суть этой задачи заключается в поиске оптимального пути, который может быть самым коротким, быстрым или экономически выгодным, проходя через промежуточные пункты один раз и возвращаясь в начальную точку. \cite{task} 
Например, это может быть оптимальный маршрут, позволяющий торговцу посетить определенные города один раз и вернуться обратно с минимальным временем, расходами или длиной пути. 
В современных условиях, когда стоимость доставки иногда сравнима с стоимостью товара, а скорость доставки играет важную роль, поиск оптимального маршрута становится критически важным.

Муравьиный алгоритм представляет собой один из эффективных методов для приближенного решения задачи коммивояжёра, а также аналогичных задач поиска маршрутов на графах. 
Этот метод основан на анализе и использовании модели поведения муравьев, которые ищут пути от колонии к источнику питания. 
Муравьиный алгоритм представляет собой оптимизацию, использующую принципы поведения муравьев для нахождения эффективных решений задачи.

Цель данной лабораторной работы заключается в описании и исследовании методов решения задачи коммивояжера.

Ниже представлены задачи, которые необходимо выполнить для достижения поставленной цели.
\begin{enumerate}
\item Описать задачу коммивояжера.
\item Описать метод решения задачи полным перебором.
\item Описать муравьиный алгоритм.
\item Разработать программу, реализующую описанные методы решения.
\item Проанализировать затраты времени работы программы и выявить их зависимость от различных параметров.
\end{enumerate}