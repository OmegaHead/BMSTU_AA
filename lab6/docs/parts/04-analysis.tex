\chapter{Аналитическая часть}

В данном разделе будут рассмотрены методы решения задачи коммивояжера: полным перебором и с использованием муравьиного алгоритма.

\section{Полный перебор}

Алгоритм полного перебора для задачи коммивояжера характеризуется высокой вычислительной сложностью ($n!$), где $n$ -- число городов. 
Суть метода заключается в исследовании всех возможных маршрутов в графе с последующим выбором наименьшего. \cite{bruteforce}
Хотя это позволяет получить оптимальное решение, временные затраты на выполнение значительны, особенно при даже небольшом количестве вершин в графе.

\section{Муравьиный алгоритм}

Метод оптимизации, известный как муравьиный алгоритм, основывается на модели поведения муравьев. 
Муравьи руководствуются своими органами чувств в процессе действия. 
Каждый муравей оставляет на своем пути феромоны, создавая таким образом след, который может быть использован другими муравьями для ориентации. 
При большом количестве муравьев наибольшее количество феромона остается на самых посещаемых путях, причем частота посещения может зависеть от длин ребер.

Идея заключается в том, что отдельный муравей ограничен в своих возможностях, поскольку способен выполнять только простые задачи. 
Однако, когда их большое количество, они могут действовать как самостоятельные вычислительные единицы. 
Муравьи взаимодействуют друг с другом, используя непрямой обмен информацией через окружающую среду посредством феромона.

Муравей обладает 3-мя свойствами:
\begin{enumerate}
    \item зрение -- муравей может видеть <<длину>> (метку) ребра / дуги и оценить привлекательность ребра;
    \item обоняние -- муравей чует концентрацию ферамона в день t на ребре / дуге;
    \item память -- у муравья есть список посещенных за текущий день t городов.
\end{enumerate}

В день $t$, находясь в городе $i$, муравей $k$ выбирает следующий город на основе следующего вероятностного правила (формула \ref{p}):
\begin{equation}
    \label{p}
    P_{ij,k}(t) = \begin{cases}
		\frac{\eta_{ij}^{\alpha}\cdot\tau_{ij}^{\beta}}{\sum_{q\notin J_k} \eta^\alpha_{iq}\cdot\tau^\beta_{iq}}, j \notin J_k \\
		0, j \in J_k
	\end{cases}
\end{equation}
где $J_k$ - список посещенных городов за текущий день; ${\eta_{ij}}$ - привлекательность ребра; ${\tau_{ij}}$ - количество ферамонов на ребре; ${\alpha}$ - коэффициент жадности решения; ${\beta}$ - коэффициент стадности; ${\alpha}$ $+$ ${\beta}$ $ = 1$; ${\alpha}$, ${\beta}$ ${\in}$ (0, 1).
При ${\alpha} = 0, {\beta} = 1$ решение стадное. При ${\alpha} = 1, {\beta} = 0$ решение жадное.

Перед наступлением нового дня феромон обновляется по формуле \ref{pheromon}:
\begin{equation}
    \label{pheromon}
    {\tau_{ij}}(t + 1) = {\tau_{ij}}(t)(1 - \rho) + {\Delta\tau_{ij}}(t), {\Delta\tau_{ij}}(t) = \sum_{k=1}^N\Delta\tau_{ij,k}(t)
\end{equation}

Модификация с элитными муравьями: перед рассветом элитный муравей (муравьи) усиливают рёбра лучшего(их) маршрута(ов).
\section*{Вывод}
В данном разделе были рассмотрены методы решения задачи коммивояжера: полным перебором и с использованием муравьиного алгоритма.
