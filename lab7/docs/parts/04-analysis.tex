\chapter{Аналитическая часть}
В данном разделе будут рассмотрены два метода поиска подстроки в строке: стандартный алгоритм и алгоритм Бойера~---~Мура.

\section{Стандартный алгоритм}

Стандартный метод поиска подстроки в строке -- это метод <<перебора>> или <<поиска вхождения>>. 
Этот метод является простым в реализации, но его эффективность по времени может быть ограничена при работе с большими объемами данных. 
Суть метода заключается в том, чтобы последовательно проверять каждую позицию в строке на предмет совпадения с искомой подстрокой~\cite{search}.

Алгоритм поиска вхождения можно описать следующим образом:

\begin{enumerate}
	\item начать сравнение искомой подстроки с каждым символом в строке, начиная с первого;
	\item если символы совпадают, перейти к следующему символу в обеих строках;
	\item если символы не совпадают, сдвинуть позицию начала сравнения в строке вправо на один символ и повторить шаг 2;
	\item повторять процесс до тех пор, пока не будет найдено точное вхождение подстроки в строку или не будет достигнут конец строки.
\end{enumerate}

Этот метод прост в понимании и реализации, но его основной недостаток заключается в том, что он может иметь квадратичную сложность времени в худшем случае. 
Например, если искомая подстрока имеет длину m, а строка имеет длину n, то сложность поиска может достигнуть O(m * n)~\cite{intro}. 

\section{Алгоритм Бойера~---~Мура}

Алгоритм Бойера~---~Мура отличается от метода <<перебора>> более сложной стратегией сдвига при несовпадении символов. 
Этот алгоритм минимизирует количество сравнений и основан на двух ключевых идеях: правиле <<хорошего суффикса>> и правиле <<плохого символа>>~\cite{search}.

Основные шаги алгоритма Бойера~---~Мура:
\begin{enumerate}
	\item предобработка паттерна (искомой подстроки):
	\begin{itemize}
		\item составить таблицу сдвигов для правила хорошего суффикса, которая определяет, куда следует сдвигаться при несовпадении с суффиксом паттерна;
		\item составить таблицу сдвигов для правила плохого символа, которая определяет, куда следует сдвигаться при несовпадении с символом в паттерне;
	\end{itemize}
	\item поиск в строке:
	\begin{itemize}
		\item начать поиск с конца паттерна;
		\item при сравнении символа паттерна с символом строки:
		\begin{itemize}
			\item если символы совпадают, продолжить сравнивать в обратном направлении;
			\item при несовпадении применить правило хорошего суффикса и правило плохого символа для определения оптимального сдвига.
		\end{itemize}
	\end{itemize}
\end{enumerate}

Алгоритм Бойера~---~Мура эффективен по времени в среднем и в худшем случае. 
Его преимущества проявляются особенно в случаях, когда в строке и подстроке много повторяющихся символов или суффиксов~\cite{intro}.

\section*{Вывод}

В данном разделе были рассмотрены два алгоритма поиска подстроки в строке: стандартный алгоритм и Бойера~---~Мура.
