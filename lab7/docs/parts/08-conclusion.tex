\chapter*{Заключение}
\addcontentsline{toc}{chapter}{Заключение}

По результатам исследований можно утверждать, что в лучшем случае реализация алгоритма Бойера~--~Мура менее эффективна по времени стандартного алгоритма, начиная с размера строки $2^{16}$.
Наилучшие показатели реализация алгоритма Бойера~--~Мура показывает в худшем случае -- она может быть эффективнее стандартного алгоритма вплоть до 10 раз.

Основной целью данной лабораторной работы было ислледование алгоритмов поиска подстроки в строке.
Ниже представлены выполненные задачи.

\begin{enumerate}[label={\arabic*)}]
    \item Приведено описание двух алгоритмов поиска подстроки в строке (стандартный алгоритм и алгоритм Бойера --- Мура).
    \item Разработано программное обеспечение, реализующее следующие алгоритмы:
    \begin{itemize}[label=---]
    \item стандартный алгоритм;
    \item алгоритм Бойера~---~Мура;
    \end{itemize}
    \item Были сравнены эффективности рассматриваемых алгоритмов по времени.
    \end{enumerate}
Цели и задачи исследования были успешно достигнуты, и полученные результаты позволяют лучше понять и оценить алгоритмы поиска подстроки в строке.