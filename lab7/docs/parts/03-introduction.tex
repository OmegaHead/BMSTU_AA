\chapter*{Введение}
\addcontentsline{toc}{chapter}{Введение}

В современном мире, где объемы данных неуклонно растут, эффективные по времени алгоритмы обработки строк являются ключевым элементом многих приложений и систем. 
Одним из важных задач в области строковых операций является поиск подстроки в строке. 
Эта проблема возникает в различных контекстах, начиная от поиска информации в текстовых документах и заканчивая обработкой данных в базах данных~\cite{intro}.

Главной целью данной лабораторной работы является исследование алгоритмов поиска подстроки в строке.

Ниже приведен набор задач, выполнение которого необходимо для достижения поставленной цели.
\begin{enumerate}[label={\arabic*)}]
\item Привести описание двух алгоритмов поиска подстроки в строке (стандартный алгоритм и алгоритм Бойера --- Мура).
\item Разработать программное обеспечение, реализующее следующие алгоритмы:
\begin{itemize}[label=---]
\item стандартный алгоритм;
\item алгоритм Бойера~---~Мура;
\end{itemize}
\item Сравнить эффективность по времени рассматриваемых алгоритмов.
\end{enumerate}