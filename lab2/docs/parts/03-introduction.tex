\chapter*{Введение}
\addcontentsline{toc}{chapter}{Введение}

Данная лабораторная работа заслуживает особого внимания, так как она фокусируется на изучении методов перемножения матриц. 
В области программирования и математики матрицы широко применяются, и их использование находит место в различных областях. 
Одним из основных применений матриц является их использование при выводе формул в физике, таких как:
\begin{itemize}
\item градиент;
\item дивергенция;
\item ротор.
\end{itemize}

Кроме того, необходимо учитывать разнообразные операции, выполняемые над матрицами, такие как сложение, возведение в степень и умножение. 
В зависимости от задачи, размеры матриц могут быть значительными, поэтому оптимизация операций с матрицами является важным аспектом программирования. 
В данной лабораторной работе будет рассмотрена оптимизация операции умножения матриц.

Главной целью данной лабораторной работы является описание, реализация и исследование алгоритмов умножения матриц.

Ниже приведен набор задач, выполнение которого необходимо для достижения поставленной цели.
\begin{enumerate}[label={\arabic*)}]
\item Провести описание двух алгоритмов умножения матриц.
\item Разработать программное обеспечение, реализующее следующие алгоритмы:
\begin{itemize}[label=---]
\item классический алгоритм умножения матриц;
\item алгоритм Винограда;
\item оптимизированный алгоритм Винограда.
\end{itemize}
\item Провести анализ временных характеристик работы программы и выявить факторы, влияющие на эффективность ее выполнения.
\item Сравнить производительность различных алгоритмов.
\end{enumerate}