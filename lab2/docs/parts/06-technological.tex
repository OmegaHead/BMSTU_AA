\chapter{Технологическая часть}

В данной главе представлены требования к программному обеспечению, описаны средства реализации, приведены листинги кода и функциональные тесты.

\section{Требования к программному обеспечению}\label{section:requirements}

Программное обеспечение должно удовлетворять следующим функциональным требованиям: на входе -- две матрицы, на выходе -- произведение этих матриц.

Программное обеспечение также должно соответствовать следующим требованиям:
\begin{itemize}[label=---]
	\item наличие пользовательского интерфейса для выбора действий;
	\item вывод результата умножения матриц
	\item предоставление функционала для измерения времени выполнения алгоритмов умножения матриц.
\end{itemize}

\section{Средства реализации}

Для разработки данной лабораторной работы был выбран язык программирования Swift \cite{swift}. 
Этот выбор обусловлен возможностью измерения процессорного времени \cite{cpu-time-measure} и соответствием с выдвинутыми техническими требованиям.

Измерение времени выполнения алгоритмов производится с использованием функции \textit{clock\_gettime()} \cite{cpu-time-measure}.

\section{Описание используемых типов данных}
При реализации алгоритмов будут использованы следующие структуры данных:
\begin{itemize}
	\item Количество строк -- целое число;
	\item Количество столбцов -- целое число;
	\item Матрица -- двумерный список целых чисел.
\end{itemize}

\section{Сведения о модулях программы}

Программа разбита на следующие модули:

\begin{itemize}
	\item \texttt{main.swift} --- точка входа в программу, где происходит вызов алгоритмов через интерфейс;
	\item \texttt{Algorithms.swift} --- содержит реализации алгоритмов умножения матриц (стандартный, алгоритм Винограда, оптимизированный алгоритм Винограда);
	\item \texttt{CPUTimeMeasure.swift} --- измеряет время работы алгоритмов с учетом заданного количества повторений;
	\item \texttt{GraphRenderer.swift} --- Строит графики для каждого из алгоритмов с учетом заданного количества повторений для каждого алгоритма;
\end{itemize}

\section{Реализация алгоритмов}

В листингах \ref{lst:standartMultiply} -- \ref{lst:optimizedVinogradMultiply} приведены реализации алгоритмов умножения матриц: стандартный алгоритм, алгоритм Винограда, оптимизированный алгоритм Винограда.

В листингах \ref{lst:inputMatrix} -- \ref{lst:outputMatrix} приведены реализации ввода и вывода матрицы.

\clearpage

\lstinputlisting[label=lst:standartMultiply,caption=Стандартный алгоритм умножения матриц, firstline=13,lastline=33]{../src/AA2/Sources/AA2/Algorithms.swift}
\clearpage

\lstinputlisting[label=lst:vinogradMultiply,caption=Алгоритм Винограда, firstline=35,lastline=67]{../src/AA2/Sources/AA2/Algorithms.swift}
\clearpage
\lstinputlisting[label=lst:vinogradMultiply2,caption=Продолжение листинга \ref{lst:vinogradMultiply}, firstline=68,lastline=77]{../src/AA2/Sources/AA2/Algorithms.swift}
\clearpage

\lstinputlisting[label=lst:optimizedVinogradMultiply,caption=Оптимизированный алгоритм Винограда, firstline=79,lastline=110]{../src/AA2/Sources/AA2/Algorithms.swift}
\clearpage
\lstinputlisting[label=lst:optimizedVinogradMultiply2,caption=Продолжение листинга \ref{lst:optimizedVinogradMultiply}, firstline=110,lastline=122]{../src/AA2/Sources/AA2/Algorithms.swift}
\clearpage

\lstinputlisting[label=lst:inputMatrix,caption=Функция ввода матрицы, firstline=34,lastline=71]{../src/AA2/Sources/AA2/main.swift}
\clearpage

\lstinputlisting[label=lst:outputMatrix,caption=Функция вывода матрицы, firstline=73,lastline=82]{../src/AA2/Sources/AA2/main.swift}
\clearpage

\clearpage

\section{Функциональные тесты}

В таблице приведены функциональные тесты для алгоритмов умножения матриц. 
Все тесты были успешно пройдены.

\begin{table}[h]
	\centering
	\small
	\caption{Тесты для умножения матриц}
	\begin{tabular}{|c|c|c|c|}
	\hline
	\textbf{Входная матрица A} & \textbf{Входная матрица B} & \textbf{Ожидаемый результат} \\
	\hline
	$A = \begin{bmatrix}
	1 & 2 \\
	3 & 4
	\end{bmatrix}$ &
	$B = \begin{bmatrix}
	5 & 6 \\
	7 & 8
	\end{bmatrix}$ &
	$C = \begin{bmatrix}
	19 & 22 \\
	43 & 50
	\end{bmatrix}$ 
	\\
	\hline
	$A = \begin{bmatrix}
	2 & 0 \\
	0 & 1
	\end{bmatrix}$ &
	$B = \begin{bmatrix}
	1 & 3 \\
	-1 & 2
	\end{bmatrix}$ &
	$C = \begin{bmatrix}
	2 & 6 \\
	-1 & 2
	\end{bmatrix}$ 
	\\
	\hline
	$A = \begin{bmatrix}
	1 & 2 & 3 \\
	4 & 5 & 6
	\end{bmatrix}$ &
	$B = \begin{bmatrix}
	7 & 8 \\
	9 & 10 \\
	11 & 12
	\end{bmatrix}$ &
	$C = \begin{bmatrix}
	58 & 64 \\
	139 & 154
	\end{bmatrix}$ 
	\\
	\hline
	$A = \begin{bmatrix}
	1 & 2 \\
	3 & 4
	\end{bmatrix}$ &
	$B = \begin{bmatrix}
	5 & 6 & 7 \\
	8 & 9 & 10 \\
	11 & 12 & 13
	\end{bmatrix}$ &
	\textbf{Ошибка} 
	\\
	\hline
	$A = \begin{bmatrix}
	2 & 0 \\
	0 & 1 & 3
	\end{bmatrix}$ &
	$B = \begin{bmatrix}
	1 & 3 \\
	-1 & 2
	\end{bmatrix}$ &
	\textbf{Ошибка} 
	\\
	\hline
	\end{tabular}
	\end{table}
	
\section*{Вывод}

В данной главе были представлены требования к программному обеспечению, описаны средства реализации, приведены листинги кода алгоритмов и функциональные тесты, подтверждающие корректность работы алгоритмов.
