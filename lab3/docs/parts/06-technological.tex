\chapter{Технологическая часть}

В данной главе представлены требования к программному обеспечению, описаны средства реализации, приведены листинги кода и функциональные тесты.

\section{Требования к программному обеспечению}\label{section:requirements}

Программное обеспечение должно удовлетворять следующим функциональным требованиям: на входе -- массив, на выходе -- остортированный массив.

Программное обеспечение также должно соответствовать следующим требованиям:
\begin{itemize}[label=---]
	\item наличие пользовательского интерфейса для выбора действий;
	\item вывод результата сортировки
	\item предоставление функционала для измерения времени выполнения алгоритмов сортировки.
\end{itemize}

\section{Средства реализации}

Для разработки данной лабораторной работы был выбран язык программирования Swift \cite{swift}. 
Этот выбор обусловлен возможностью измерения процессорного времени \cite{cpu-time-measure} и соответствием с выдвинутыми техническими требованиям.

Измерение времени выполнения алгоритмов производится с использованием функции \textit{clock\_gettime()} \cite{cpu-time-measure}.

\section{Сведения о модулях программы}

Программа разбита на следующие модули:

\begin{itemize}
	\item \texttt{main.swift} --- точка входа в программу, где происходит вызов алгоритмов через интерфейс;
	\item \texttt{Algorithms.swift} --- содержит реализации алгоритмов сортировки (блинная, поразрядная, бинарным деревом);
	\item \texttt{CPUTimeMeasure.swift} --- измеряет время работы алгоритмов с учетом заданного количества повторений;
	\item \texttt{GraphRenderer.swift} --- Строит графики для каждого из алгоритмов с учетом заданного количества повторений для каждого алгоритма;
\end{itemize}

\section{Реализация алгоритмов}

В листингах \ref{lst:pancake} -- \ref{lst:tree} приведены реализации алгоритмов сортировки: блинная, поразрядная, бинарным деревом.

В листинге \ref{lst:input} приведена реализации ввода массива.

\clearpage

\lstinputlisting[label=lst:pancake,caption=Блинная сортировка, firstline=70,lastline=79]{../src/AA3/Sources/AA3/Algorithms.swift}
\lstinputlisting[label=lst:findMaxIndex,caption=Функция поиска индекса максимального элемента, firstline=45,lastline=58]{../src/AA3/Sources/AA3/Algorithms.swift}
\lstinputlisting[label=lst:flip,caption=Функция переворачивания массива, firstline=60,lastline=68]{../src/AA3/Sources/AA3/Algorithms.swift}
\clearpage

\lstinputlisting[label=lst:radixSort,caption=Поразрядная сортировка, firstline=115,lastline=123]{../src/AA3/Sources/AA3/Algorithms.swift}
\lstinputlisting[label=lst:getMax,caption=Функция поиска максимума, firstline=81,lastline=89]{../src/AA3/Sources/AA3/Algorithms.swift}
\clearpage

\lstinputlisting[label=lst:countSort,caption=Сортировка подсчетом, firstline=91,lastline=113]{../src/AA3/Sources/AA3/Algorithms.swift}
\clearpage

\lstinputlisting[label=lst:treeSort,caption=Сортировка бинарным деревом, firstline=125,lastline=139]{../src/AA3/Sources/AA3/Algorithms.swift}
\clearpage

\lstinputlisting[label=lst:tree,caption=Бинарное дерево, firstline=9,lastline=39]{../src/AA3/Sources/AA3/Algorithms.swift}
\clearpage

\lstinputlisting[label=lst:input,caption=Функция ввода массива, firstline=30,lastline=45]{../src/AA3/Sources/AA3/main.swift}
\clearpage

\clearpage

\section{Функциональные тесты}

В таблице приведены функциональные тесты для алгоритмов сортировки. 
Все тесты были успешно пройдены.

\begin{table}[h]
	\centering
	\small
	\caption{Тесты алгоритмов сортировки}
	\begin{tabular}{|c|c|c|}
	\hline
	\textbf{Входной массив} & \textbf{Ожидаемый результат} \\
	\hline
	[1, 2, 3, 4, 5, 6, 7] &
	[1, 2, 3, 4, 5, 6, 7]
	\\
	\hline
	[7, 6, 5, 4, 3, 2, 1]&
	[1, 2, 3, 4, 5, 6, 7] 
	\\
	\hline
	[1, 5, 3, 7, 6, 2, 4]&
	[1, 2, 3, 4, 5, 6, 7] 
	\\
	\hline
	[1, 1, 1, 1, 5, 5, 1, 2, 2]&
	[1, 1, 1, 1, 1, 2 ,2, 5, 5]
	\\
	\hline
	\end{tabular}
	\end{table}
	
\section*{Вывод}

В данной главе были представлены требования к программному обеспечению, описаны средства реализации, приведены листинги кода алгоритмов и функциональные тесты, подтверждающие корректность работы алгоритмов.
