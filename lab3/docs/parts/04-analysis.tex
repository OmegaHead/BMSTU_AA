\chapter{Аналитическая часть}
В данном разделе мы рассмотрим три алгоритма сортировок: блинная, поразрядная, бинарным деревом.

\section{Блинная сортировка}

Блинная сортировка (англ. pancake sorting) \cite{pancake-sort} -- это алгоритм сортировки, который использует операции переворачивания элементов в массиве для достижения упорядочивания элементов в правильном порядке. 
Суть этого алгоритма заключается в пошаговом перемещении наибольшего элемента массива в правильное положение, а затем уменьшении размера обрабатываемой части массива и повторении этого процесса до тех пор, пока весь массив не будет упорядочен.

Ниже представлены основные шаги блинной сортировки.
\begin{enumerate}
\item Начать с полного массива, который нужно отсортировать.
\item Найти индекс максимального элемента в текущей части массива.
\item Перевернуть массив так, чтобы максимальный элемент переместился на верхнюю позицию (максимальный элемент становится первым).
\item Перевернуть массив снова, чтобы максимальный элемент оказался в правильном положении (он становится последним элементом).
\item Уменьшить размер текущей части массива, и перейти к следующему наибольшему элементу, повторяя шаги 2-4, пока весь массив не будет упорядочен.
\end{enumerate}

Блинная сортировка обычно рассматривается как не самый эффективный алгоритм сортировки из-за своей высокой временной сложности в худшем случае $O(N^3)$, но он обладает простой и наглядной логикой. 
Этот алгоритм иногда используется для обучения и демонстрации основных понятий сортировки и манипуляции с элементами массива.

\section{Поразрядная сортировка}

Поразрядная сортировка (англ. Radix Sort) \cite{radix-sort} -- это алгоритм сортировки, который основывается на разрядах (цифрах) чисел. 
Он работает для целых чисел или других данных, которые можно разделить на разряды. 
Алгоритм выполняет сортировку путем поочередного рассмотрения чисел по разрядам, начиная с самого младшего разряда и двигаясь к старшему разряду. 
Внутри каждого прохода по разрядам используется какой-либо стабильный алгоритм сортировки, обычно сортировка подсчетом (counting sort) или сортировка вставками (insertion sort).

Ниже представлены основные шаги поразрядной сортировки.
\begin{enumerate}
\item Определить максимальное количество разрядов в числах, которые нужно отсортировать.
\item Начать с самого младшего разряда и перейти к старшим разрядам поочередно.
\item Для каждого разряда (начиная с младшего) выполнить сортировку элементов с учетом этого разряда, используя стабильный сортировочный алгоритм. Например, можно применить сортировку подсчетом или сортировку вставками для элементов в текущем разряде.
\item Повторять шаг 3 для каждого разряда, двигаясь от младшего к старшему.
\item После обработки всех разрядов массив будет отсортирован.
\end{enumerate}

Поразрядная сортировка может быть эффективной, особенно когда числа имеют ограниченное количество разрядов, и все разряды равнозначны. 
Однако она может быть не так эффективной, если числа имеют разное количество разрядов, и в худшем случае она имеет временную сложность $O(n*k)$, где n - количество элементов, а k - количество разрядов.

Этот алгоритм часто используется для сортировки целых чисел, строк и других данных, которые можно представить как последовательность разрядов.

\section{Сортировка бинарным деревом}
Сортировка бинарным деревом \cite{bst-sort}, также известная как сортировка двоичным деревом поиска (Binary Search Tree Sort), это алгоритм сортировки, который использует бинарное дерево поиска для упорядочивания элементов. 
Этот алгоритм начинает с пустого бинарного дерева \cite{bst-sort} и поочередно вставляет элементы в дерево так, чтобы сохранить порядок сортировки. 
Затем, когда все элементы вставлены в дерево, они извлекаются в упорядоченной последовательности с помощью обхода в порядке возрастания (in-order traversal) бинарного дерева.

Ниже представлены основные шаги сортировки бинарным деревом.
\begin{enumerate}
\item Начать с пустого бинарного дерева.
\item Последовательно вставить все элементы из исходного массива в бинарное дерево. 
При вставке элемента следует учитывать его значение и соблюдать порядок сортировки. 
Это означает, что элементы с меньшими значениями должны быть помещены в левое поддерево, а элементы с большими значениями - в правое поддерево.
\item После вставки всех элементов в бинарное дерево выполнить обход в порядке возрастания (in-order traversal) этого дерева. 
Обход возвращает элементы в упорядоченной последовательности.
\item Элементы, извлеченные в результате обхода, образуют упорядоченный массив, который представляет собой отсортированную версию исходного массива.
\end{enumerate}
Сортировка бинарным деревом имеет среднюю временную сложность $O(n*log(n))$, где n - количество элементов, однако в худшем случае (если элементы добавляются в дерево в уже упорядоченном порядке) может иметь временную сложность $O(n^2)$, что делает ее менее эффективной, чем некоторые другие алгоритмы сортировки. 
Кроме того, сортировка бинарным деревом требует дополнительной памяти для хранения структуры дерева.

\section*{Вывод}
В данном разделе мы рассмотрели алгоритмы сортировки: блинную, поразрядную, бинарным деревом.
Рассмотренные сортировки следует реализовать, изучить трудоемкость и проанализировать время и память для выполнения.