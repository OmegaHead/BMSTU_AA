\chapter*{Введение}
\addcontentsline{toc}{chapter}{Введение}

В данной лабораторной работе будет проведен анализ сортировок.

Сортировка - это переупорядочивание некой последовательности или кортежа в определенном порядке. 
Эта операция является важной частью обработки структурированных данных. 
Упорядоченное расположение элементов позволяет более эффективно работать с данными, особенно при поиске нужных элементов.

Существует множество алгоритмов сортировки, но каждый из них включает в себя следующие основные элементы:
\begin{itemize}
\item сравнение элементов, определяющее их порядок;
\item перестановка для изменения местоположения элементов;
\item алгоритм, который использует сравнение и перестановку для сортировки данных.
\end{itemize}

Каждый алгоритм обладает своими преимуществами, и его эффективность оценивается на основе ответов на следующие вопросы:
\begin{itemize}
\item какая средняя скорость сортировки этим алгоритмом;
\item каковы лучший и худший случаи сортировки;
\item проявляется ли "естественное" поведение алгоритма, т.е. увеличивается ли скорость сортировки с увеличением упорядоченности массива;
\item является ли алгоритм стабильным, то есть сохраняет ли он порядок элементов с одинаковыми значениями.
\end{itemize}

Цель данной лабораторной работы заключается в описании и исследовании трудоемкости алгоритмов сортировки.

Ниже представлены задачи, которые необходимо выполнить для достижения поставленной цели.
\begin{enumerate}
\item Разработать программное обеспечение, реализующее следующие алгоритмы сортировки:
\begin{itemize}
\item Блинная;
\item Поразрядная;
\item Бинарным деревом.
\end{itemize}
\item Оценить трудоемкость этих алгоритмов сортировки.
\item Измерить время выполнения алгоритмов.
\item Проанализировать затраты времени работы программы и выявить их зависимость от различных характеристик.
\end{enumerate}