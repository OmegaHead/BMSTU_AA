\chapter{Аналитическая часть}

\section{Расстояние Левенштейна}

Расстояние Левенштейна, также известное как редакционное расстояние, представляет собой метрику, которая позволяет измерить различие между двумя последовательностями символов. 
Оно определяется как минимальное количество операций редактирования, таких как вставка (I), замена (R) и удаление (D), необходимых для превращения одной строки в другую. 
Каждая из этих операций имеет свою стоимость:

\begin{enumerate}[label=\arabic*)]
	\item $w(a, b)$ - стоимость замены символа $a$ на символ $b$;
	\item $w(\lambda, b)$ - стоимость вставки символа $b$;
	\item $w(a, \lambda)$ - стоимость удаления символа $a$.
\end{enumerate}

В данной работе мы будем рассматривать стандартную стоимость, где каждая из этих операций имеет стоимость 1:

\begin{itemize}[label=---]
	\item $w(a, b) = 1$, если $a \neq b$, иначе замена не требуется;
	\item $w(\lambda, b) = 1$;
	\item $w(a, \lambda) = 1$.
\end{itemize}

Для обозначения совпадения символов используется символ M (match) с нулевой стоимостью: $w(a, a) = 0$.

Мы также вводим функцию $D(i, j)$, которая представляет собой расстояние Левенштейна между подстроками $S_1[1...i]$ и $S_2[1...j]$.

Расстояние Левенштейна между двумя строками $S_1$ и $S_2$ длиной $M$ и $N$ соответственно может быть вычислено с использованием рекуррентной формулы:

\begin{equation}
	\label{eq:L}
	D(i, j) =
	\begin{cases}
		0, &\text{i = 0, j = 0}\\
		i, &\text{j = 0, i > 0}\\
		j, &\text{i = 0, j > 0}\\
		\min \begin{cases}
			D(i, j - 1) + 1,\\
			D(i - 1, j) + 1,\\
			D(i - 1, j - 1) +  m(S_{1}[i], S_{2}[j]), \\
		\end{cases}
		&\text{i > 0, j > 0}
	\end{cases}
\end{equation}

Где функция $m(a, b)$ определена как:

\begin{equation}
	\label{eq:m}
	m(a, b) = \begin{cases}
		0 &\text{если a = b,}\\
		1 &\text{иначе.}
	\end{cases}
\end{equation}

\section{Нерекурсивный алгоритм для расстояния Левенштейна}

Рекурсивная реализация алгоритма Левенштейна может быть неэффективной по времени при больших значениях $M$ и $N$, из-за повторных вычислений. 
Для оптимизации можно использовать итеративную реализацию с использованием матрицы размером $(N + 1) \times (M + 1)$ для хранения промежуточных значений $D(i,j)$. 
Значение в ячейке $[i, j]$ будет представлять собой значение $D(S1[1...i], S2[1...j])$. 
Начальная ячейка заполняется нулем, а затем остальные значения заполняются согласно формуле (\ref{eq:L}).

Однако стоит отметить, что матричный алгоритм может быть малоэффективным по памяти, особенно при больших значениях $M$ и $N$, так как множество промежуточных значений $D(i,j)$ должно быть хранено в памяти. 
Для оптимизации по памяти можно использовать рекурсивный алгоритм с кешем, который хранит значения $D(i,j)$, вычисленные на предыдущей итерации, и значения, вычисленные на текущей итерации.

\section{Расстояние Дамерау~---~Левенштейна}

Расстояние Дамерау~---~Левенштейна, названное в честь ученых Фредерика Дамерау и Владимира Левенштейна, является мерой разницы между двумя строками символов. 
Оно определяется как минимальное количество операций вставки, удаления, замены и транспозиции (перестановки двух соседних символов), необходимых для превращения одной строки в другую. 
Это является модификацией расстояния Левенштейна, в которое добавляется операция транспозиции $T$ (transposition).

Расстояние Дамерау~---~Левенштейна можно вычислить с использованием рекуррентной формулы:

\begin{equation}
	\label{eq:DL}
	D(i, j) = 
	\begin{cases}
		0, &\text{i = 0, j = 0,}\\
		i, &\text{j = 0, i > 0,}\\
		j, &\text{i = 0, j > 0,}\\
		\min \begin{cases}
			D(i, j - 1) + 1,\\
			D(i - 1, j) + 1,\\
			D(i - 1, j - 1) +\\
			+ m(S_{1}[i], S_{2}[j]),\\
			D(i - 2, j - 2) + 1,\\
		\end{cases}
		& \begin{aligned}
			& \text{если i > 1, j > 1},\\
			& S_{1}[i] = S_{2}[j-1],\\
			& S_{1}[i-1] = S_{2}[j],\\
		\end{aligned}\\
		\min \begin{cases}
			D(i, j - 1) + 1,\\
			D(i - 1, j) + 1,\\
			D(i - 1, j - 1) +\\
			+m(S_{1}[i], S_{2}[j]),\\
		\end{cases}
		 & \text{иначе.}
	\end{cases}
\end{equation}

\section{Рекурсивный алгоритм для расстояния Дамерау~---~Левенштейна}

Рекурсивный алгоритм для расстояния Дамерау~---~Левенштейна реализует формулу (\ref{eq:DL}) и позволяет вычислить расстояние между двумя строками. 
Он обладает следующими характеристиками:

\begin{enumerate}
	\item Для перевода из пустой строки в пустую строку требуется ноль операций.
	\item Для перевода из пустой строки в строку $a$ требуется $|a|$ операций.
	\item Для перевода из строки $a$ в пустую строку также требуется $|a|$ операций.
	\item Для перевода из строки $a$ в строку $b$ требуется выполнить некоторое количество операций вставки, удаления, замены и транспозиции. Порядок выполнения операций не имеет значения. 
\end{enumerate}

Минимальная стоимость преобразования достигается путем выбора наименьшей из перечисленных выше операций.

\section{Рекурсивный алгоритм для расстояния Дамерау~---~Левенштейна с кэшированием}

Рекурсивная реализация алгоритма Дамерау~---~Левенштейна может быть неэффективной по времени, особенно при больших значениях $M$ и $N$, из-за повторных вычислений значений расстояний между подстроками. 
Для оптимизации можно использовать кэш, который хранит значения $D(i,j)$, вычисленные на предыдущей итерации, а также значения, вычисленные на текущей итерации.

\section*{Вывод}

В данном разделе мы рассмотрели алгоритмы для вычисления расстояний Левенштейна и Дамерау~---~Левенштейна. 
Эти алгоритмы могут быть реализованы как рекурсивно, так и итеративно. 
Они позволяют измерить различие между двумя строками символов, что может быть полезно во многих задачах, таких как автокоррекция и поиск похожих строк.
