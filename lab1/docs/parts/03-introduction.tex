\chapter*{Введение}
\addcontentsline{toc}{chapter}{Введение}

В данной лабораторной работе предстоит исследовать понятие расстояния Левенштейна \cite{levenshtein}. 
Это метрика, которая показывает наименьшее количество операций (вставка, удаление, замена), требуемых для превращения одной строки в другую. 
Эта метрика играет важную роль в определении сходства между двумя строками.

Интерес к расстоянию Левенштейна был впервые вызван советским математиком Владимиром Левенштейном в 1965 году, когда он исследовал последовательности символов «0» и «1». 
Позже эта задача была обобщена для произвольного алфавита и получила его имя.

Расстояние Левенштейна имеет широкое применение в теории информации и компьютерной лингвистике. 
Его используют для решения следующих задач:
\begin{itemize}[label=---]
\item исправление опечаток в словах (в поисковых системах, базах данных, при вводе текста и при автоматическом распознавании отсканированных текстов или речи);
\item сравнение текстовых файлов с помощью утилиты diff;
\item сравнение геномов, хромосом и белков в области биоинформатики.
\end{itemize}

Главной целью этой лабораторной работы является описание и исследование алгоритмов, связанных с расстоянием Левенштейна и Дамерау~---~Левенштейна.
Ниже представлены задачи, которые необходимо выполнить для достижения этой цели.
\begin{enumerate}[label={\arabic*)}]
\item Подробно описать алгоритмы для вычисления расстояний Левенштейна и Дамерау~---~Левенштейна.
\item Разработать программное обеспечение, реализующее следующие алгоритмы:
\begin{itemize}[label=---]
\item итеративный алгоритм для вычисления расстояния Левенштейна;
\item итеративный алгоритм для вычисления расстояния Дамерау~---~Левенштейна;
\item рекурсивный алгоритм для вычисления расстояния Дамерау~---~Левенштейна;
\item рекурсивный алгоритм с использованием кэширования для вычисления расстояния Дамерау~---~Левенштейна.
\end{itemize}
\item Выбрать инструменты для измерения процессорного времени выполнения реализаций алгоритмов.
\item Провести анализ затрат времени и памяти для различных реализаций алгоритмов и выявить факторы, влияющие на эти затраты.
\end{enumerate}