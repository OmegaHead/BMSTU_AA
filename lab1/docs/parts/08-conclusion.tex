\chapter*{Заключение}
\addcontentsline{toc}{chapter}{Заключение}

Результаты данного исследования позволяют сделать несколько важных выводов. 
В частности, было установлено, что время выполнения алгоритмов вычисления расстояний Левенштейна и Дамерау~---~Левенштейна возрастает в геометрической прогрессии при увеличении длины строк. 
Среди всех реализаций алгоритмов наилучшие показатели по времени демонстрируют матричная реализация алгоритма Дамерау~---~Левенштейна и его рекурсивная версия с использованием кеша.

Основной целью данной лабораторной работы было изучение и описание особенностей задач динамического программирования, а именно алгоритмов Левенштейна и Дамерау~---~Левенштейна. 
Ниже представлены выполненные задачи.

\begin{enumerate}
\item Проведено подробное описание алгоритмов для вычисления расстояния Левенштейна и Дамерау~---~Левенштейна.
\item Разработано программное обеспечение, реализующее следующие алгоритмы:
\begin{itemize}[label=---]
\item Нерекурсивный метод вычисления расстояния Левенштейна.
\item Нерекурсивный метод вычисления расстояния Дамерау~---~Левенштейна.
\item Рекурсивный метод вычисления расстояния Дамерау~---~Левенштейна.
\item Рекурсивный метод вычисления расстояния Дамерау~---~Левенштейна с использованием кеша.
\end{itemize}
\item Выбраны инструменты для измерения процессорного времени выполнения реализаций алгоритмов.
\item Проведен анализ временных и памятных затрат программы, чтобы выявить влияющие на них характеристики и факторы.
\end{enumerate}

Цели и задачи исследования были успешно достигнуты, и полученные результаты позволяют лучше понять и оценить эффективность различных методов вычисления расстояний Левенштейна и Дамерау~---~Левенштейна.